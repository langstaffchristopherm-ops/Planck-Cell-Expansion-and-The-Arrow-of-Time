\section*{Notation}

\noindent\textbf{Conventions.}  
Time and entropy advance in discrete, \emph{dimensionless ticks}.  
A single primitive entropy unit is created per tick, giving
\[
  \Delta S = \Delta \tau.
\]
Physical mapping is applied only when necessary:
\[
  S_{\mathrm{phys}} = k_B\,S, \qquad
  \tau_s = t_P\,\tau.
\]
Fundamental constants: speed of light \(c\), reduced Planck constant \(\hbar\), and Boltzmann constant \(k_B\).

\smallskip
\noindent\textit{Native--SI correspondence.}  
One hop per tick implies \(\mathrm{d}a/\mathrm{d}\tau = 1\).  
Choosing a length per hop \(\ell_0\) and a time per tick \(t_0\) yields
\[
  c = \frac{\ell_0}{t_0}.
\]

\bigskip
\noindent\textbf{Core quantities.}
\begin{description}[leftmargin=2.4em,labelsep=0.8em]
  \item[\(k\)] Tick index (integer). The scale factor \(a(k)\) is defined with respect to ticks; baseline \(a(k)\propto k^{1/3}\).
  \item[\(\Delta S,\,\Delta\tau\)] Primitive entropy and proper-time tick \emph{counts} (dimensionless) satisfying \(\Delta S=\Delta\tau\).
  \item[\(z\)] Observed redshift, with \(a(k)/a(k_0) = (1+z)^{-1}\).
  \item[\(\Lambda\)] Scale-similarity parameter.
  \item[\(c\)] Speed of light (constant tick-propagation rate).
\end{description}

\noindent\textbf{Graph and geometric parameters.}
\begin{description}[leftmargin=2.4em,labelsep=0.8em]
  \item[\(d_G(x,y)\)] Graph (hop) distance between vertices \(x\) and \(y\).
  \item[\(\ell_N\)] Typical interior hop length for a sample of size \(N\).
  \item[\(\varepsilon_N,\,\delta_N\)] Mesh non-uniformity and directional-bias parameters (\(\to 0\) as \(N\to\infty\)).
  \item[\(\eta_N\)] Combined small parameter: \(\eta_N := C_1\varepsilon_N + C_2\delta_N.\)
  \item[\(\mathrm{distortion}(N)\)] Bilipschitz distortion of the graph metric relative to the ambient metric:
  \[
    \mathrm{distortion}(N)
      = \sup_{x,y}
        \max\!\left\{
          \frac{d_G(x,y)}{\|x-y\|},\,
          \frac{\|x-y\|}{d_G(x,y)}
        \right\}-1.
  \]
\end{description}

\noindent\textbf{Expansion-specific quantities.}
\begin{description}[leftmargin=2.4em,labelsep=0.8em]
  \item[\(a(k)\)] Scale factor as a function of tick index; baseline \(a(k)\propto k^{1/3}\).
  \item[\(H\)] Tick-based expansion rate, \(H = \frac{1}{a}\frac{\mathrm{d}a}{\mathrm{d}\tau}\) (per proper tick).
  \item[\(\Omega_\cdot\)] Fractional density parameters when a fluid picture is invoked (optional, if used).
  \item[\(z\)] Observed redshift, \(a(k)/a(k_0) = (1+z)^{-1}\).
  \item[\(\Lambda\)] Scale-similarity parameter connecting epochs or energy scales.
  \item[\(c\)] Constant light speed (fixed tick-propagation rate).
\end{description}

\noindent Tick–redshift mapping and derived expansion law:
\[
  k(z) = k_0(1+z)^{-3}, \qquad
  H_{\mathrm{S/t}}(z)
    = \frac{1}{3}\,\frac{\mathrm{d}\ln k(z)}{\mathrm{d}\tau}
    = -\,\frac{1}{1+z}\,\frac{\mathrm{d}z}{\mathrm{d}\tau}.
\]

\bigskip
\noindent\textbf{Distances and acoustic rulers.}  
Standard distance measures apply with \(H_{\mathrm{S/t}}(z)\) replacing the cosmological Hubble parameter \cite{hogg1999}:
\begin{align*}
D_C(z) &= c\int_0^{z}\frac{\mathrm{d}z'}{H_{\mathrm{S/t}}(z')}, &
D_A(z) &= \frac{D_M(z)}{1+z}, &
D_L(z) &= (1+z)\,D_M(z), \\[2pt]
r_d &= \int_{z_d}^{\infty}\frac{c_s(z)}{H_{\mathrm{S/t}}(z)}\,\mathrm{d}z, &
\theta_\star &= \frac{r_s(z_\star)}{D_A(z_\star)}.
\end{align*}
For flat geometry, \(D_M = D_C\).  
If early microphysics is standard,
\[
  c_s(z) = \frac{c}{\sqrt{3\,[1+R(z)]}},
  \qquad
  R(z) = \frac{3\rho_b}{4\rho_\gamma}
\]
\cite{husugiyama1996,eisenstein1998};  
otherwise substitute the S/t channel sound speed.  
For practical \(z_\star\) and \(z_d\), we use the Hu–Sugiyama and Eisenstein–Hu fits \cite{husugiyama1996,eisenstein1998}.

\bigskip
\noindent\textbf{Growth and clustering.}  
Linear growth follows \cite{dodelson2003}:
\[
  \ddot{\delta}
  + 2H_{\mathrm{S/t}}\,\dot{\delta}
  - 4\pi G\,\rho_m\,\delta = 0,
\]
with \(\rho_m\) derived from stiffness clocks via \(m = \hbar\omega_0/c^2\) and comoving counts.  
Amplitude observables (\(\sigma_8\), lensing) thus probe both \(H_{\mathrm{S/t}}\) and the mass assignment law.

\bigskip
\noindent\textbf{Quick symbol summary.}
\begin{description}[leftmargin=2.4em,labelsep=0.8em]
  \item[\(\ell_P\)] Planck-cell edge or hop length.
  \item[\(t_P\)] Planck tick (one global update interval).
  \item[\(c=\ell_P/t_P\)] Causal speed cap (one hop per tick).
  \item[\(k\)] Discrete tick index (global ledger step).
  \item[\(R(k)\)] Front radius after \(k\) ticks.
  \item[\(N(k)\)] Number of active/born cells after \(k\) ticks.
  \item[\(L\)] Continuum baseline distance.
  \item[\(d_G\)] Graph (hop) distance.
  \item[\(T(L)\)] Transit time over baseline \(L\).
  \item[\(v_g\)] Group or front speed.
  \item[\(E,\,\hbar\)] Energy and reduced Planck constant.
  \item[\(\Delta\phi\)] Accumulated phase along the path.
  \item[\(\lambda\)] Wavelength of the probe signal.
\end{description}
